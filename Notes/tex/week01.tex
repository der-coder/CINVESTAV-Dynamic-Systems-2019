\chapter{Signals \& Systems}

\section{What is a system?}

A system is defined as a grouping of elements to be 
analysed together. They can be categorised as linear or 
non-linear, depending on the equations used to 
describe them.
Linear systems are considered to be idealised systems, 
whilst non-linear systems are those representing real-world 
conditions.

Systems may also be categorised based on the
\emph{order} of their differential equations. Some examples 
of categorised systems are shown in table \ref{table: 
examples of systems}.


\begin{table}[h!]
\centering
\begin{tabular}{c|cc}
 & Linear & Non-Linear\\
 \hline
1st order & RC circuit & Population growth\\
2nd order & Spring Mass Damper & Pendulum\\
3rd order & - & Chaotic Systems\\
... & ...  & ... \\
Nth order & Wave Equation & General Relativity\\
\hline
\end{tabular}
\label{table: examples of systems}
\caption{Examples of linear and non-linear systems.}
\end{table}

\section{Linear Systems}
A system is considered linear if they meet the following 
two principles: 

\begin{itemize}
 \item Proportionality. Given an input $t$, the system will 
return an output $f(t)$, and if given an input $\alpha t$ 
the system will return an output $f(\alpha t) = \alpha 
f(t)$.
 \item Superposition. Given inputs $t_1$ and $t_2$, the 
system will return outputs $f(t_1)$ and $f(t_2)$ 
respectively. If given an input $t_1 + t_2$ the system will 
return an output $f(t_1 + t_2) = f(t_1) + f(t_2)$.
\end{itemize}

\subsection{Exercise 1: Linear systems}
Given the following equations, determine if they belong to 
linear systems or not.

\textbf{Case 1:} $f(s) = 3s$. Evaluating the function to 
test proportionality, we obtain the following:

\begin{equation*}
 \begin{split}
  f(2s) & =3(2s)\\
  & = 6 s\\
  2 f(s) &= 2 (3s)\\
  &= 6s\\
  f(2s)&=2f(s)
 \end{split}
\end{equation*}

For the superposition principle, we have:
\begin{equation*}
 \begin{split}
  f(s_1) & =3s_1\\
  f(s_2) & =3s_2\\
  f(s_1+s_2) & =3(s_1+s_2)\\
   &= 3s_1 + 3s_2\\
  f(s_1 + s_2) & = f(s_1) + f(s_2)
 \end{split}
\end{equation*}

We conclude that case 1 is linear.

\vspace*{1cm}

An equation can be tested to meet both principles 
simultaneously. Let $s=\alpha(s_1+s_2)$ and $f(s)$ be the 
output of the system. A system is linear if the following 
condition is met:

\begin{equation}
  f(\alpha(s_1+s_2))  = \alpha f(s_1) + \alpha f(s_2)
  \label{eq: proportionality and superposition}
\end{equation}

Evaluating \ref{eq: proportionality and superposition} for 
case 1 as an example:

\begin{equation*}
\begin{split}
 f(\alpha (s_1 + s_2)) & = 3 (\alpha (s_1 + s_2))\\
 & = 3 \alpha s_1 + 3 \alpha s_2\\
 f(\alpha s) & = 3 \alpha s \\
 f(\alpha (s_1 + s_2)) & = \alpha f(s_1) + \alpha f(s_2)
\end{split}
\end{equation*}

\textbf{Case 2:} $3s + 1$

Case 2 is non-linear.

\vspace*{1cm}

\textbf{Case 3:} $0.5 \cos (0.1 s) $

Case 3 is non-linear.

\vspace*{1cm}

\textbf{Case 4:} $1.2 \mathrm{e}^{0.1 s}$

Case 4 is non-linear.

\vspace*{1cm}

\textbf{Case 5:} $\int\limits_0^t s(t)dt$

Case 5 is linear.

\vspace*{1cm}

\textbf{Case 6:} $\frac{ds(t)}{dt}$

Case 6 is linear.

\vspace*{1cm}
