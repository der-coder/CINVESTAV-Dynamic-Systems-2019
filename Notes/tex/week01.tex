\chapter{Signals \& Systems}

\section{What is a system?}

A system is defined as a grouping of elements to be 
analysed together. They can be categorised as linear or 
non-linear, depending on the equations used to 
describe them.
Linear systems are considered to be idealised systems, 
whilst non-linear systems are those representing real-world 
conditions.

Systems may also be categorised based on the
\emph{order} of their differential equations. Some examples 
of categorised systems are shown in table \ref{table: 
examples of systems}.


\begin{table}[h!]
\centering
\begin{tabular}{c|cc}
 & Linear & Non-Linear\\
 \hline
1st order & RC circuit & Population growth\\
2nd order & Spring Mass Damper & Pendulum\\
3rd order & - & Chaotic Systems\\
... & ...  & ... \\
Nth order & Wave Equation & General Relativity\\
\hline
\end{tabular}
\label{table: examples of systems}
\caption{Examples of linear and non-linear systems.}
\end{table}

\section{Linear Systems}
A system is considered linear if they meet the following 
two principles: 

\begin{itemize}
 \item Proportionality. Given an input $t$, the system will 
return an output $f(t)$, and if given an input $\alpha t$ 
the system will return an output $f(\alpha t) = \alpha 
f(t)$.
 \item Superposition. Given inputs $t_1$ and $t_2$, the 
system will return outputs $f(t_1)$ and $f(t_2)$ 
respectively. If given an input $t_1 + t_2$ the system will 
return an output $f(t_1 + t_2) = f(t_1) + f(t_2)$.
\end{itemize}

\subsection{Exercise: Linear systems}
Given the following equations, determine if they belong to 
linear systems or not.

\textbf{Case 1:} $f(s) = 3s$. Evaluating the function to 
test proportionality, we obtain the following:

\begin{equation*}
 \begin{split}
  f(2s) & =3(2s)\\
  & = 6 s\\
  2 f(s) &= 2 (3s)\\
  &= 6s\\
  f(2s)&=2f(s)
 \end{split}
\end{equation*}

For the superposition principle, we have:
\begin{equation*}
 \begin{split}
  f(s_1) & =3s_1\\
  f(s_2) & =3s_2\\
  f(s_1+s_2) & =3(s_1+s_2)\\
   &= 3s_1 + 3s_2\\
  f(s_1 + s_2) & = f(s_1) + f(s_2)
 \end{split}
\end{equation*}

We conclude that case 1 is linear.

% \vspace*{1cm}

An equation can be tested to meet both principles 
simultaneously. Let $s=\alpha(s_1+s_2)$ and $f(s)$ be the 
output of the system. A system is linear if the following 
condition is met:

\begin{equation}
  f(\alpha(s_1+s_2))  = \alpha f(s_1) + \alpha f(s_2)
  \label{eq: proportionality and superposition}
\end{equation}

Evaluating \ref{eq: proportionality and superposition} for 
case 1 as an example:

\begin{equation*}
\begin{split}
 f(\alpha (s_1 + s_2)) & = 3 (\alpha (s_1 + s_2))\\
 & = 3 \alpha s_1 + 3 \alpha s_2\\
 f(\alpha s) & = 3 \alpha s \\
 f(\alpha (s_1 + s_2)) & = \alpha f(s_1) + \alpha f(s_2)
\end{split}
\end{equation*}

\textbf{Case 2:} $f(s) = 3s + 1$.

\begin{equation*}
 \begin{split}
  f(\alpha s) & = 3 \alpha  s +1\\
  \alpha f(s) & = 3 \alpha s + \alpha\\
  f(\alpha s) & \neq \alpha f(s)\\
  f(s_1) + f(s_2) & = 3 s_1 + 3 s_2 + 2\\
  f(s_1 + s_2) & = 3 (s_1+s_2) + 1\\
  f(s_1 +s_2) & \neq f(s_1) + f(s_2)
 \end{split}
\end{equation*}


Case 2 is non-linear.

% \vspace*{1cm}

\textbf{Case 3:} $f(s) = 0.5 \cos (0.1 s) $

\begin{equation*}
 \begin{split}
  f(\alpha s) & = 0.5 \cos (0.1 \alpha s)\\
  \alpha f(s) & = 0.5 \alpha \cos (0.1 s)\\
  f(\alpha s) & \neq \alpha f(s)\\
  f(s_1) + f(s_2) & = 0.5 \cos (0.1 s_1) + 0.5 \cos (0.1 
s_2) \\
  f(s_1 + s_2) & = 0.5 \cos (0.1 s_1 + 0.1 s_2)\\
  f(s_1 + s_2) & \neq f(s_1) + f(s_2)
 \end{split}
\end{equation*}


Case 3 is non-linear.

% \vspace*{1cm}

\textbf{Case 4:} $f(s) = 1.2 \mathrm{e}^{0.1 s}$

\begin{equation*}
 \begin{split}
f(\alpha (s_1+s_2))  & = 1.2 \mathrm{e}^{0.1 \alpha (s_1 + 
s_2)}\\
\alpha f(s_1) & = 1.2 \alpha \mathrm{e}^{0.1 s_1}\\
\alpha f(s_1) + \alpha f(s_2) & = 1.2 \alpha 
(\mathrm{e}^{0.1 s_1} + \mathrm{e}^{0.1 s_2})\\
f(\alpha (s_1+s_2))  & \neq \alpha f(s_1) + \alpha f(s_2)
 \end{split}
\end{equation*}


Case 4 is non-linear.

% \vspace*{1cm}

\textbf{Case 5:} $f(s) = \int\limits_0^t s(t)dt$
\begin{equation*}
 \begin{split}
f(\alpha (s_1 + s_2))  & = \alpha \int\limits_0^t 
\Bigl( s_1(t)+s_2(t) \Bigr)  dt\\
& = \alpha \int\limits_0^t 
s_1(t)dt + \alpha \int\limits_0^t 
s_2(t)dt\\
\alpha f(s_1) + \alpha f(s_2) & = \alpha \int\limits_0^t 
s_1(t)dt + \alpha \int\limits_0^t 
s_2(t)dt\\
f(\alpha (s_1+s_2))  & = \alpha f(s_1) + \alpha f(s_2)
 \end{split}
\end{equation*}

Case 5 is linear.

% \vspace*{1cm}

\textbf{Case 6:} $f(s) = \frac{ds(t)}{dt}$

\begin{equation*}
 \begin{split}
f(\alpha (s_1 + s_2))  & =  \alpha 
\frac{d \Bigl (s_1(t)+s_2(t) \Bigr)}{dt}\\
& = \alpha \frac{d s_1(t)}{dt} + \alpha \frac{d 
s_2(t)}{dt} \\
\alpha f(s_1) + \alpha f(s_2) & = \alpha \frac{d s_1(t)}{dt} 
+ \alpha \frac{d s_2(t)}{dt} \\
f(\alpha (s_1+s_2))  & = \alpha f(s_1) + \alpha f(s_2)
 \end{split}
\end{equation*}

Case 6 is linear.

Note that for all equations $\alpha \neq 1$ because using 
the identity is not valid for proving compliance.

\subsection{Order of a system}

The order of a system is dependent on the highest order of 
derivatives in the equations that describe it. A system 
with no derivative terms is a \emph{zero order} system, a 
system with a differential equation of order 1 is a 
\emph{first-order} system, and so on.

Revisiting the cases shown in Exercise 1, we can classify 
the systems depending on their order as well. Table 
\ref{table: exercise 1 order} contains the classification 
of each case.

\begin{table}[t]
\centering
\begin{tabular}{c|cc}
Equation & Type & Order\\
\hline
$3s$ & Linear & 0\\
$3s + 1$ & Non-linear & 0\\
$0.5 \cos (0.1 s)$ & Non-Linear & 0\\
$1.2 \mathrm{e}^{0.1s}$ & Non-Linear & 0\\
$\int\limits_0^{t} s(t)dt$ & Linear & 0\\
$\frac{d s(t)}{dt}$ & Linear & 1\\
\hline
\end{tabular}
\label{table: exercise 1 order}
\caption{Classification of systems.}
\end{table}

\section{Example: Spring-Mass-Damper System}

The Spring-Mass-Damper System (abbreviated SMD) is the most 
commonly used abstraction for systems. Depending on the 
initial conditions of the systems, it may or may not be 
linear. An example is presented below.

Given the forces interacting in the Free Body Diagram, the 
system's equation is obtained as follows:

\begin{equation}
 \begin{split}
  \sum F & = 0\\
  F_s(t) + F_d(t) - F(t) & = 0\\
 \end{split}
 \label{eq: sum of forces}
\end{equation}

Recall the equations for springs and dampers, substituting 
them in \ref{eq: sum of forces}.

\begin{equation*}
 \begin{split}
  F_s & = b \dot{x}\\
  F_d & = k x(t)\\
  b \dot{x} + k x(t) & = F(t)
 \end{split}
\end{equation*}

Rewriting the equation into the form $\dot x + f(t) x(t) = 
g(t)$.

\begin{equation}
 \dot{x} + \frac{k}{b} x(t) = \frac{1}{b} F(t)
 \label{eq: system equation}
\end{equation}

\subsection{Linear System Case}

Solving the system via Laplace Transforms to obtain an 
equation for x(t). Consider $x(0) = 0$ as the initial 
condition.

\begin{equation}
 \begin{split}
  \mathfrak{L} \Bigl \{ \dot{x} + \frac{k}{b} x(t)  = 
\frac{1}{b} F(t) \Bigr \} & \xrightarrow{} s \mathcal{X} - 
x(0) + \frac{k}{b} \mathcal{X} = \frac{1}{b} \mathcal{F}
 \end{split}
 \label{eq: general laplace transform}
\end{equation}

Evaluating $x(0)=0$ gives us the system equation in the 
frequency domain.

\begin{equation}
 s \mathcal{X} + \frac{k}{b} \mathcal{X} = \frac{1}{b}
\mathcal{F}
\label{eq: laplace transform}
\end{equation}

Solve \ref{eq: laplace transform} for $\mathcal{X}$ and 
then obtain the inverse Laplace transform of the equation.

\begin{equation*}
 \begin{split}
  \mathcal{X} (s + \frac{k}{b}) & = \frac{1}{b} 
\mathcal{F}\\
\mathcal{X} & = \frac{1}{b} \frac{1}{s + \frac{k}{b}}
\mathcal{F}\\
 \end{split}
\end{equation*}


\begin{equation}
 \begin{split}
  \mathfrak{L}^{-1} \Bigl \{ \mathcal{X}  = \frac{1}{b} 
\frac{1}{s + \frac{k}{b}}
\mathcal{F}   \Bigr\} & \xrightarrow{} x(F(t)) = 
\frac{1}{b} \mathrm{e}^{-\frac{k}{b}t} F(t)
 \end{split}
 \label{eq: linear system equation}
\end{equation}

Testing the principles of proportionality and superposition 
on \ref{eq: linear system equation} we have

\begin{equation*}
 \begin{split}
  x(\alpha (F_1 + F_2)) & = \frac{\alpha}{b} 
\mathrm{e}^{-\frac{k}{b}t} (F_1 + F_2)\\
\alpha x(F_1) & = \frac{\alpha}{b} 
\mathrm{e}^{-\frac{k}{b}t} F_1\\
\alpha x(F_1) + \alpha x(F_2) & = \frac{\alpha}{b} 
\mathrm{e}^{-\frac{k}{b}t} (F_1 + F_2)\\
x(\alpha (F_1 + F_2)) & = \alpha x(F_1) + \alpha x(F_2)
 \end{split}
\end{equation*}

We confirm that the system is linear for an initial 
condition of $x(0)=0$.

\section{Homework}

\subsection{Non-Linear Case}

Given the conditions $x(0) = 0.5$ and $F(t) = F$, we 
evaluate the Laplace transform of \ref{eq: system equation} 
and obtain

\begin{equation}
  s \mathcal{X} - 0.5 + \frac{k}{b} \mathcal{X} = 
\frac{1}{b} \frac{F}{s}
\end{equation}

Solving for $\mathcal{X}$.

\begin{equation}
 \mathcal{X} = \frac{F}{b}
\frac{1}{s(s+\frac{k}{b})} + \frac{0.5}{s+\frac{k}{b}} 
\label{eq: non-linear system equation laplace}
\end{equation}

Using partial fractions, find an equivalent expression for 
\ref{eq: non-linear system equation laplace} .

\begin{equation*}
 \begin{split}
  \frac{1}{s(s+\frac{k}{b})}  & = \frac{A}{s} + 
\frac{B}{s+\frac{k}{b}}\\
A & = \frac{b}{k}\\
B &= -\frac{b}{k}\\
 \end{split}
\end{equation*}

\begin{equation}
 \frac{1}{s(s+\frac{k}{b})}   = \frac{1}{s} \frac{b}{k} - 
\frac{b}{k}
\frac{1}{s+\frac{k}{b}}
\label{eq: partial fractions}
\end{equation}

Substitute \ref{eq: partial fractions} in \ref{eq: 
non-linear system equation laplace} and then obtain the 
inverse Laplace transform of the equation.

\begin{equation*}
 \begin{split}
   \mathcal{X} & = \frac{F}{b}
\Bigl ( \frac{1}{s} \frac{b}{k} - 
\frac{b}{k}
\frac{1}{s+\frac{k}{b}} \Bigr) + \frac{0.5}{s+\frac{k}{b}} 
\\
& = \frac{F}{k} \frac{1}{s}  - 
\frac{F}{k}
\frac{1}{s+\frac{k}{b}} + \frac{0.5}{s+\frac{k}{b}} 
\\
 \end{split}
\end{equation*}

\begin{equation}
 \mathfrak{L}^{-1} \Bigl \{ \mathcal{X} \Bigr \} 
\xrightarrow{} x(t) = \frac{F}{k} - \frac{F}{k} 
\mathrm{e}^{-\frac{k}{b}t} + 0.5 \mathrm{e}^{-\frac{k}{b}t}
\label{eq: non-linear system equation}
\end{equation}

Test the principles of proportionality and superposition on 
\ref{eq: non-linear system equation}.

\begin{equation*}
 \begin{split}
 x(\alpha t) & = \frac{F}{k} - \frac{F}{k} 
\mathrm{e}^{-\alpha \frac{k}{b}t} + 0.5 
\mathrm{e}^{-\alpha \frac{k}{b}t}\\
 \alpha x(t) &= \alpha \frac{F}{k} - \alpha \frac{F}{k} 
\mathrm{e}^{-\frac{k}{b}t} + 0.5 \alpha 
\mathrm{e}^{-\frac{k}{b}t}\\
x(\alpha t) & \neq \alpha x(t)\\
  x( t_1 + t_2) & = \frac{F}{k} - \frac{F}{k} 
\mathrm{e}^{-\frac{k}{b}(t_1+t_2)} + 0.5 
\mathrm{e}^{-\frac{k}{b}(t_1+t_2)}\\
x(t_1) +  x(t_2) & = 2 \frac{F}{k} - 
\frac{F}{k} \Bigl ( 
\mathrm{e}^{- \frac{k}{b}t_1} -\mathrm{e}^{- \frac{k}{b}t_2} 
\Bigr ) + 0.5 
\Bigl (\mathrm{e}^{- \frac{k}{b}t_1}
 +
\mathrm{e}^{- \frac{k}{b}t_2} \Bigr) \\
x(t_1 + t_2) & \neq x(t_1) + x(t_2)\\
 \end{split}
\end{equation*}

We conclude that the system in Non-linear for the given 
initial conditions.
